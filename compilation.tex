\documentclass[8pt,a4paper,compress,handout]{beamer}

\usepackage{/home/siyer/lib/slides}

\title{Compilation}
\date{}

\begin{document}
\begin{frame}
\vfill
\titlepage
\end{frame}

\begin{frame}
\frametitle{Outline}
\tableofcontents
\end{frame}

\section{Compilers}
\begin{frame}[fragile]
\pause

A compiler is a program that translates a source program written in a high-level programming language such as Java, C\lstinline{#} or C, into a target program in a lower level language such as machine code

\begin{center}
\includegraphics[scale=0.7]{{figures/figure01.01}.jpg}
\end{center}

\pause
\bigskip

A programming language is an artificial language in which a programmer writes a program to control the behavior of a machine, particularly a computer

\pause
\bigskip

Like a natural language, in a programming language, one describes
\begin{itemize}
\item The tokens (aka lexemes)

\item The syntax of programs and language constructs such as classes, methods, statements and expressions

\item The meaning (aka semantics) of the various constructs
\end{itemize}
\end{frame}

\section{Why Study Compilers?}
\begin{frame}[fragile]
\pause


\end{frame}

\section{The Phases of Compilation}
\begin{frame}[fragile]
\pause


\end{frame}

\section{Overview of the \protect \jmm to JVM Compiler}
\begin{frame}[fragile]
\pause


\end{frame}

\section{The \protect \jmm Compiler Source Tree}
\begin{frame}[fragile]
\pause


\end{frame}
\end{document}
